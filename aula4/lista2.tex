\documentclass[10pt,a4paper]{report}

\usepackage[brazil]{babel}
\usepackage[T1]{fontenc}
\usepackage[latin1]{inputenc}
\usepackage{amsmath,amsfonts}
\usepackage{amssymb}
\usepackage{tikz}
\usepackage{latexsym}
\usepackage{pgf}

\newcounter{conta}
 
\begin{document}
 
 \hfill DECSI - UFOP \\
{\it Grupo de Estudos de Programa\c{c}\~ao Funcional}
 \hfill $\mbox{2}^{\mbox{\underline{o}}}$ semestre de 2013 \\
Professor: \parbox[t]{14cm}{Rodrigo Geraldo Ribeiro \\
                     e-mail: rodrigogribeiro@decsi.ufop.br}
 
\noindent {\bf Lista de Exerc\'icios 2} \hfill {\bf Tema: Fun\c{c}\~oes de Ordem Superior}

\hfill

Em an\'alise num\'erica, a \emph{regra de Horner} \'e um algoritmo eficiente para a avalia\c{c}\~ao / 
representa\c{c}\~ao de polin\^omios.

Dado o polin\^omio:
\begin{equation*}
   p(x) = \sum_{k=0}^{n}a_{k}x^{k} = a_{n}x^{n} + a_{n - 1}x^{n - 1} + ... + a_{1}x^{1} + a_{0}
\end{equation*}
onde $a_{0}, ..., a_{n}$ s\~ao n\'umeros reais, podemos orden\'a-lo usando uma
ordem decrescente de expoentes. Uma vez que os expoentes s\~ao apresentados em ordem decrescente, podemos representar
um polin\^omio utilizando apenas uma lista que cont\^em seus coeficientes. Desta maneira, o seguinte tipo Haskell pode
ser utilizado para representar um polin\^omio:

\begin{verbatim}
   type Polynomial = [Int]
\end{verbatim}
isto \'e, um polin\^omio \'e representado como uma lista de seus coeficientes em ordem decrescente de expoentes.
Se um valor $a_{i}$ est\'a na $i$-\'esima
posi\c{c}\~ao da lista, ent\~ao o coeficiente $a_{i}$ corresponde ao termo $x^{i}$. Isto \'e, a $i$-\'esima 
posi\c{c}\~ao da lista representa o termo correspondente ao expoente $i$. Considere o seguinte polin\^omio:
\begin{equation*}
   p(x) = 4x^{3} + 2x^{2} - 5
\end{equation*}
Este seria representado como:
\begin{verbatim}
   px = [4, 2, 0, -5]
\end{verbatim}
Cabe ressaltar que caso o polin\^omio n\~ao possua um termo correspondente a um determinado expoente, este \'e
 representado com o coeficiente $0$. No polin\^omio $4x^{3}+2x^{2}-5$, n\~ao existe termo para o expoente $1$, e,
 portanto, este \'e representado utilizando o valor $0$ como seu coeficiente.
 
Baseado no que foi apresentado, desenvolva o que se pede a seguir. Para cada item, \'e apresentado um exemplo 
envolvendo o polin\^omio $4x^{3} + 2x^{2} - 5$, isto \'e, \texttt{[4,2,0,-5]}.

\begin{enumerate}
    \item Desenvolva a fun\c{c}\~ao:
    \begin{verbatim}
		grade :: Polynomial -> Int
    \end{verbatim}
    que retorna o maior expoente do polin\^omio fornecido como par\^ametro. Exemplo:
    \begin{verbatim}
        grade [4, 2, 0, -5] = 3
    \end{verbatim}
    \item Para somar dois polin\^omios basta somar os coeficientes de termos de mesmo expoente. O objetivo deste
          exerc\'icio \'e implementar a fun\c{c}\~ao:
    \begin{verbatim}
		(.+.) :: Polynomial -> Polynomial -> Polynomial
    \end{verbatim}
    que recebe dois polin\^omios como par\^ametros e retorna como resultado a soma destes. Por\'em, nem sempre os dois
    polin\^omios a serem somados possuem o mesmo grau (maior expoente). Como exemplo, considere
    \begin{center}\[
	    \begin{array}{lcl}
    	   p_{1}(x) & = & 4x^{3} + 2x^{2} - 5\\
	       p_{2}(x) & = & 6x^{2} + 1
    	\end{array} \]   
    \end{center}    
    Portanto, temos que: $ p_{1}(x) + p_{2}(x) = 4x^{3} + 8x^{2} - 4$. Um inconveniente deste fato, \'e que a 
    representa\c{c}\~ao de polin\^omios n\~ao est\'a normalizada, isto \'e, ambos os polin\^omios a serem somados 
    n\~ao possuem o mesmo grau. No exemplo anterior, temos que:
    \begin{verbatim}
		p1x = [4, 2, 0, -5]
		p2x = [6, 0, 1]
    \end{verbatim} 
    A vers\~ao normalizada destes polin\^omios (isto \'e, ambos possuindo o mesmo grau) seria:
    \begin{verbatim}
		p1x = [4, 2, 0, -5]
		p2x = [0, 6, 0, 1]
    \end{verbatim} 
    Observe que a \'unica altera\c{c}\~ao realizada foi acrescentar no segundo polin\^omio o coeficiente $0$, para
    que ambos possu\'issem termos de expoente $3$.
    \begin{enumerate}
    	\item Desenvolva a fun\c{c}\~ao:
    	\begin{verbatim}
			normalize :: Polynomial -> Polynomial -> (Polynomial, Polynomial)
    	\end{verbatim}
    	que recebe como par\^ametro dois polin\^omios possivelmente n\~ao normalizados e retorna um par contendo
    	os dois polin\^omios fornecidos como par\^ametros normalizados. Exemplo:
    	\begin{verbatim}
normalize [4, 2, 0, -5] [6, 0, 1] = ([4,2,0,-5], [0,6,0,1])
    	\end{verbatim}
    	\item Utilizando a fun\c{c}\~ao \texttt{normalize}, desenvolvida no item anterior, implemente a fun\c{c}\~ao
	    \begin{verbatim}
		(.+.) :: Polynomial -> Polynomial -> Polynomial
    	\end{verbatim}   
    	que soma dois polin\^omios fornecidos como argumento. Exemplo:
    	\begin{verbatim}
[4,2,0,-5] .+. [0,6,0,1] = [4, 8, 0, -4]    	   
    	\end{verbatim}
    \end{enumerate}
    \item Seja $p(x)$ o seguinte polin\^omio
\begin{equation*}
   p(x) = \sum_{k=0}^{n}a_{k}x^{k} = a_{n}x^{n} + a_{n - 1}x^{n - 1} + ... + a_{1}x^{1} + a_{0}
\end{equation*}    
    A derivada de $p(x)$, $\frac{\displaystyle{d}}{\displaystyle{dx}}(p(x))$, \'e definida como:
\begin{equation*}
   \frac{\displaystyle{d}}{\displaystyle{dx}}(p(x)) = \sum_{k=1}^{n}k\times (a_{k}x^{k - 1}) =\\
   n\times a_{n}x^{n - 1} + (n - 1)\times a_{n - 1}x^{(n - 1) - 1} + ... + 1\times a_{1}x^{0}
\end{equation*}    
   Como exemplo, considere o polin\^omio $4x^{3} + 2x^{2} - 5$. A derivada deste \'e igual a:
   \begin{equation*}
       \frac{\displaystyle{d}}{\displaystyle{dx}}(4x^{3} + 2x^{2} - 5) = \\
       3\times 4x^{2} + 2 \times 2x^{1} = \\
       12 x^{2} + 4x
   \end{equation*}
   Com base no apresentado, desenvolva a fun\c{c}\~ao:
   \begin{verbatim}
derivative :: Polynomial -> Polynomial
   \end{verbatim}
   que dado um polin\^omio como entrada, retorna a derivada deste. Exemplo:
   \begin{verbatim}
derivative [4, 2, 0, -5] = [12, 4, 0]
   \end{verbatim}
   \item A representa\c{c}\~ao de Horner pode ser utilizada para obter um algoritmo eficiente para calcular o 
   valor de um polin\^omio para um determinado valor $x_{0}$. Seja $p(x)$ o polin\^omio:
\begin{equation*}
   p(x) = \sum_{k=0}^{n}a_{k}x^{k} = a_{n}x^{n} + a_{n - 1}x^{n - 1} + ... + a_{1}x^{1} + a_{0}
\end{equation*} 
e $x_{0}$ um inteiro qualquer. Ent\~ao $p(x_{0})$ \'e igual a:
\begin{equation*}
 p(x_{0}) = a_{0} + x \times (a_{1} + x \times (a_{2} + ... + x\times (a_{n - 1} + x\times a_{n}\underbrace{) ... ))}_{\text{n par\^entesis}}
\end{equation*}
Com base no apresentado, desenvolva a fun\c{c}\~ao:
\begin{verbatim}
   eval :: Int -> Polynomial -> Int
\end{verbatim}
que recebe um valor inteiro e um polin\^omio e retorna como resultado o valor deste polin\^omio para o inteiro
 fornecido como par\^ametro. Exemplo:
\begin{verbatim}
  eval 2 [4 , 2, 0, -5] = 35
\end{verbatim}
\end{enumerate}
\end{document}
