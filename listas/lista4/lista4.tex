\documentclass[11pt,a4paper]{article}

\usepackage[brazil]{babel}
\usepackage[T1]{fontenc}
\usepackage[latin1]{inputenc}
\usepackage{amsmath,amsfonts}
\usepackage{bussproofs}
\usepackage{amssymb}
\usepackage{tikz}
\usepackage{latexsym}
\usepackage{pgf}
\usepackage[pdftex]{hyperref}

 
\begin{document}
\noindent Curso de Sistemas de Informa\c{c}\~ao / Engenharia de Computa\c{c}\~ao\\
UFOP \\
{\it Grupo de Estudos de Programa\c{c}\~ao Funcional}\\
Professor: \parbox[t]{14cm}{Rodrigo Geraldo Ribeiro \\
                     e-mail: rodrigo@decsi.ufop.br}
  

\section{Xeque-Mate!}

O xadrez \'e um dos jogos de estrat\'egia mais antigos de toda a hist\'oria.
Neste trabalho voc\^e dever\'a implementar fun\c{c}\~oes para realizar o 
processamento de arquivos PGN --- Portable Game Notation --- que s\~ao normalmente 
utilizados para a descri\c{c}\~ao de jogos de xadrez. Maiores detalhes podem 
ser encontrados em:
\begin{center}
	\href{http://en.wikipedia.org/wiki/Portable_Game_Notation}{http://en.wikipedia.org/wiki/Portable$\_$Game$\_$Notation}
\end{center}
Diversos softwares permitem utilizar este formato para ``salvar'' o progresso
de um determinado jogo de xadrez ou mesmo para permitir a visualiza\c{c}\~ao
``passo-a-passo'' de uma partida descrita neste formato.

N\~ao existem descri\c{c}\~oes formais deste formato, portanto adotaremos 
uma descri\c{c}\~ao aproximada que deve ser sufuciente para que as 
fun\c{c}\~oes implementadas processem um grande n\'umero de arquivos PGN
dispon\'iveis na internet.

Para a implementa\c{c}\~ao deste trabalho, n\~ao \'e necess\'ario um 
conhecimento pr\'evio de Xadrez. Caso voc\^e n\~ao conhe\c{c}a 
absolutamente nada sobre as regras de Xadrez, uma boa refer\^encia \'e:
\begin{center}
	\href{http://en.wikipedia.org/wiki/Chess}{http://en.wikipedia.org/wiki/Chess}
\end{center}
que dever\'a ser suficiente para voc\^e conhecer sobre como s\~ao 
identificadas cada uma das posi\c{c}\~oes do tabuleiro, cada uma das pe\c{c}as
e seus respectivos movimentos. 

Para este trabalho, voc\^e dever\'a utilizar a biblioteca de \emph{parsing}
presente no arquivo \texttt{ParseLib.hs}. N\~ao ser\'a necess\'ario voc\^e
implement\'a-la, pois esta estar\'a dispon\'ivel na p\'agina da disciplina 
(arquivo \texttt{ParseLib.hs}). Adicionalmente, ser\'a fornecida uma 
implementa\c{c}\~ao parcial que voc\^e dever\'a utilizar no desenvolvimento
de sua solu\c{c}\~ao (arquivo \texttt{Chess.hs}). Sinta-se livre para 
adicionar quantos m\'odulos julgar necess\'ario em seu programa. 
Ao contr\'ario do trabalho anterior, menos c\'odigo ser\'a fornecido. Isso
se deve ao fato que grande parte das fun\c{c}\~oes que voc\^e dever\'a 
codificar depender\~ao dos tipos de dados que ser\~ao definidos por voc\^e
para representar a \'arvore de sintaxe da gram\'atica que especifica 
arquivos PGN. 

\section{Especifica\c{c}\~ao}

A especifica\c{c}\~ao do trabalho ser\'a feita por uma descri\c{c}\~ao 
passo-a-passo. Para isso, primeiramente descreveremos a estrutura de
um movimento e em seguida especificaremos como dever\'a ser a estrutura
de um arquivo PGN por completo.

\subsection{Movimentos}\label{san}

A nota\c{c}\~ao padr\~ao para descri\c{c}\~ao de movimentos em jogos de 
xadrez \'e denominada \textit{SAN --- Standard Algebraic Notation}, que
\'e descrita pela seguinte gram\'atica livre de contexto:

\begin{tabular}{lcl}
	\textit{move} & $\rightarrow$ & \textit{piece disambiguation capture square promotion check}\\
	              & $|$           & \verb|O-O-O| \\
	              & $|$           & \verb|O-O| \\
	\textit{piece} & $\rightarrow$ & \texttt{N $|$ R $|$ B $|$ Q $|$ K $|\,\lambda$}\\
	\textit{disambiguation} & $\rightarrow$ & \textit{file $|$ rank $|$ square $|\,\lambda$}\\
	\textit{capture} & $\rightarrow$ & \texttt{x} $|\,\lambda$\\
	\textit{square} & $\rightarrow$ & \textit{file rank} \\
	\textit{promotion} & $\rightarrow$ & \texttt{=} \textit{piece} $|\,\lambda$\\
	\textit{check} & $\rightarrow$ & \texttt{+} $|$ $\#$ $|\,\lambda$\\
	\textit{file} & $\rightarrow$ & \texttt{a $|$ b $|$ c $|$ d $|$ e $|$ f $|$ g $|$ h}\\
	\textit{rank} & $\rightarrow$ & \texttt{1 $|$ 2 $|$ 3 $|$ 4 $|$ 5 $|$ 6 $|$ 7 $|$ 8}\\
\end{tabular}
Terminais s\~ao descritos usando fonte \texttt{typewriter} e n\~ao-terminais 
em \textit{it\'alico}. Um movimento pode ser regular ou especial. Movimentos
regulares s\~ao descritos por uma sequ\^encia indicando a pe\c{c}a a ser 
movida, uma sequ\^encia para evitar ambiguidades entre pe\c{c}as de uma 
mesma categoria fornecendo dados sobre a posi\c{c}\~ao inicial do 
movimento, uma indica\c{c}\~ao se o movimento resultou em uma captura, 
a posi\c{c}\~ao de destino do movimento descrito, informa\c{c}\~ao sobre
uma poss\'ivel promo\c{c}\~ao de um pe\~ao e se este movimento resultou
em um xeque ou xeque-mate. Existem dois movimentos especiais, denominados
``roque'', onde o rei e uma das torres s\~ao movidas simultaneamente. O 
roque do lado da rainha \'e especificado por \texttt{O-O-O} (composto por
letras \texttt{O} e \texttt{-}) e o do lado do rei por \texttt{O-O}.

A pe\c{c}a movida \'e indicada por uma letra mai\'uscula. A letra \texttt{N}
representa o cavalo (kNight), \texttt{B} bispo (Bishop), 
\texttt{R} torre (Rook), \texttt{Q} rainha (Queen) e \texttt{K} rei (King).
N\~ao \'e utilizada nenhuma letra para representar o movimento de um pe\~ao.

Na maioria dos casos, as regras do jogo e / ou a situa\c{c}\~ao do tabuleiro 
determinam unicamente, dado uma posi\c{c}\~ao de destino, qual pe\c{c}a est\'a
sendo movida. Se este n\~ao \'e o caso, alguma informa\c{c}\~ao sobre a posi\c{c}\~ao
inicial pode ser fornecida, por meio de um file (coluna), rank (linha) ou uma 
posi\c{c}\~ao espec\'ifica do tabuleiro. Um file consiste da coluna onde uma pe\c{c}a
est\'a localizada e \'e destrito por uma letra de \texttt{a} a \texttt{h}. Um rank
\'e a linha onde a pe\c{c}a est\'a localizada e \'e descrita por um n\'umero de 1 a 8. 

Se durante um movimento h\'a uma captura, isso \'e indicado por um \texttt{x} entre
uma especifica\c{c}\~ao de ambiguidade e a posi\c{c}\~ao de destino.

Se um pe\~ao alcan\c{c}ar a linha ``base'' do openente, ele pode ser ``promovido'' para
outra pe\c{c}a. Isto \'e indicado pelo s\'imbolo \texttt{=} e a letra da pe\c{c}a para a
qual o pe\~ao ser\'a promovido.

Quando um movimento resulta em um xeque ou em um xeque-mate, isto \'e indicado pelo 
acr\'escimo dos s\'imbolos \texttt{+} ou um \texttt{\#}, respectivamente, a descri\c{c}\~ao
do movimento.

\textbf{N\~ao \'e permitido nenhum tipo de espa\c{c}os em uma string que especifica um movimento!}

\begin{enumerate}
	\item (\emph{trivial}) --- Defina tipos de dados Haskell para descrever a sintaxe de um movimento. D\^e o nome
	      \texttt{Move} para este tipo de dados (como especificado no c\'odigo fornecido).
	      Especifique uma cl\'ausula \texttt{deriving Eq} e uma inst\^ancia de \texttt{Show}
	      para este tipo de dados que permita a impress\~ao de um movimento representado por um
	      valor do tipo que voc\^e definiu.
	\item (\emph{m\'edio}) --- Defina um parser
	\begin{center}
		\texttt{parseMove :: Parser Char Move}
	\end{center}
	que realiza o parsing de um \'unico movimento. Seja \texttt{m} um valor qualquer de tipo
	\texttt{Move} sua fun\c{c}\~ao \texttt{parseMove} deve ser tal que
	\begin{center}
		\texttt{run parseMove (show m) == m}
	\end{center}
	deve retornar \texttt{True}. Para testar seu parser, utilize as seguintes strings:
	\begin{itemize}
		\item ``\texttt{e4}''
		\item ``\texttt{Nf3xg5+}''
		\item ``\texttt{Qa8=N}''
	\end{itemize}
	O \'ultimo exemplo, apesar de aceito pela gram\'atica (e por consequ\^encia, por seu parser), n\~ao \'e
	um movimento v\'alido em um jogo de xadrez, j\'a que somente pe\~oes podem ser promovidos (neste movimento,
	\'e feita a promo\c{c}\~ao de uma rainha). Uma promo\c{c}\~ao \'e v\'alida se um pe\~ao de cor preta for 
	movido para a linha 1 ou se um pe\~ao branco for movido para a linha 8.
	\item (\emph{f\'acil}) --- Escreva uma fun\c{c}\~ao
	\begin{center}
		\texttt{promotionCheck :: Move -> Bool}
	\end{center}
	que verifica se um movimento possui uma promo\c{c}\~ao ilegal. Isto \'e ele deve retornar \texttt{True} sempre
	que um movimento n\~ao possuir uma promo\c{c}\~ao ou possuir uma promo\c{c}\~ao v\'alida.
\end{enumerate}

\subsection{Sintaxe de Arquivos PGN}

A sintaxe de um arquivo PGN \'e dada pela seguinte gram\'atica livre de contexto:

\begin{tabular}{rcl}
	\textit{pgn} & $\rightarrow$ & \textit{game pgn $|\,\lambda$}\\
	\textit{game} & $\rightarrow$ & \textit{tags movetext}\\
	\textit{tags} & $\rightarrow$ & \textit{tag tags $|\,\lambda$}\\
	\textit{tag} & $\rightarrow$ & \texttt{[}\textit{name value}\texttt{]}\\
	\textit{name} & $\rightarrow$ & \textit{symbol}\\
	\textit{value} & $\rightarrow$ & \textit{string}\\
	\textit{movetext} & $\rightarrow$ & \textit{elements termination}\\
	\textit{elements} & $\rightarrow$ & \textit{element elements}\\
					  & $|$           & \textit{variation elements}\\
					  & $|$           & $\lambda$\\ 
	\textit{element}  & $\rightarrow$ & \textit{numopt move nagopt}\\
	                  & $|$           & $\lambda$\\		
	\textit{variation} & $\rightarrow$ & \texttt{(}\textit{elements}\texttt{)}\\
	\textit{termination} & $\rightarrow$ & \texttt{1-0}\\
						 & $|$           & \texttt{0-1}\\
						 & $|$           & \texttt{1/2-1/2}\\
						 & $|$           & \texttt{*}\\	
	\textit{numopt}       & $\rightarrow$ & \textit{number}\\
						 & $|$           & $\lambda$\\	  
	\textit{nagopt}       & $\rightarrow$ & \textit{nag}\\
	                     & $|$           & $\lambda$\\
	\textit{number}      & $\rightarrow$ & \textit{natural dots}\\
	\textit{dots}        & $\rightarrow$ & \texttt{.} \textit{dots}\\
						 & $|$           & $\lambda$\\
	\textit{nag}         & $\rightarrow$ & \texttt{\$} $digit^{*}$\\
\end{tabular}

Diferentes n\~ao terminais na gram\'atica anterior s\~ao seperados por
um n\'umero arbitr\'ario de espa\c{c}os. 

Os componentes l\'exicos s\~ao especificados pela seguinte gram\'atica:

\begin{tabular}{rcl}
	\textit{symbol} & $\rightarrow$ & (\textit{letter $|$ digit})(\textit{letter $|$ digit} $|$ \texttt{\_} $|$ 
	\texttt{\#} $|$ \texttt{=} $|$ \texttt{:} $|$ \texttt{-} $|$ \texttt{/})$^{*}$\\
	\textit{string} & $\rightarrow$ & \texttt{``} (\textit{quoted $|$ char})$^{*}$ \texttt{''}\\
	\textit{quoted} & $\rightarrow$ & \verb|\|\textit{char}\\
					& $|$           & \verb|\\|\\
					& $|$           & \verb|\"|\\
	\textit{letter} & $\rightarrow$ & qualquer letra mai\'uscula ou minin\'uscula.\\
	\textit{digit}  & $\rightarrow$ & qualquer d\'igito.\\
	\textit{char}   & $\rightarrow$ & qualquer caractere imprim\'ivel exceto \verb|"| e \verb|\|\\
\end{tabular}

Nenhum espa\c{c}o em branco \'e permitido nas produ\c{c}\~oes da espeficica\c{c}\~ao l\'exica.

Informalmente, todo arquivo PGN consiste de uma sequ\^encia de jogos. Todo jogo, por sua vez, consiste 
de duas partes: tags e movimentos. Tags s\~ao uma lista de pares nome-valor entre colchetes. A lista 
de movimentos e assim o um jogo como um todo \'e sempre encerrado por um indicador do resultado deste jogo,
que pode ser \texttt{1-0} (pe\c{c}as brancas venceram), \texttt{0-1} (pe\c{c}as pretas venceram), 
\texttt{1/2-1/2} (empate) ou \texttt{*} (resultado desconhecido). 

Cada movimento \'e introduzido por um n\'umero opcional, que por sua vez pode ser seguido por um n\'umero
arbitr\'ario de pontos. Um movimento propriamente dito \'e descrito usando a nota\c{c}\~ao \emph{SAN} 
(introduzida na se\c{c}\~ao \ref{san}). Um movimento pode possuir um valor n\'umerico opcional, denominado NAG
--- Numeric Annotation Glyph --- que nada mais \'e que um n\'umero natural que indica algum tipo de informa\c{c}\~ao
(muitas vezes subjetiva) sobre um determinado movimento (por exemplo, se este foi bom ou ruim). O formato PGN pode
tamb\'em descrever varia\c{c}\~oes de um jogo (algo simular a `` o que aconteceria se...''). Tais varia\c{c}\~oes
consistem de uma sequ\^encia de movimentos entre par\^enteses. Na gram\'atica apresentada anteriormente, note a 
diferen\c{c}a entre par\^enteses que pertencem a linguagem de arquivos PGN (denotados por \texttt{()} em fonte 
\texttt{typewriter}) de par\^enteses que s\~ao utilizados na nota\c{c}\~ao que descreve gram\'aticas (que s\~ao 
par\^enteses escritos em fonte normal.).

Um s\'imbolo \'e introduzido por uma letra ou d\'igito, e ent\~ao seguido por um n\'umero arbitr\'ario de letras
, d\'igitos ou alguns caracteres espec\'ificos. S\'imbolos s\~ao usados para descrever os nomes das tags. Strings
s\~ao envolvidos por aspas duplas ``'' e s\~ao utilizadas para descrever valores de tags. Um NAG \'e introduzido
por um s\'imbolo \texttt{\$} seguido de um n\'umero qualquer de d\'igitos. 

\begin{itemize}
	\item[4 .] (\emph{f\'acil}) --- Defina tipos de dados Haskell para representar a sintaxe de arquivos PGN.
	\item[5 .] (\emph{dif\'icil}) --- Defina uma fun\c{c}\~ao
	\begin{center}
		\texttt{parsePGN :: Parser Char PGN}
	\end{center}
	que realiza o parsing de um arquivo PGN completo. Observer que n\~ao \'e obrigat\'orio 
	que voc\^e realize o parsing diretamente a partir de uma string. Voc\^e pode optar por
	contruir um analisar l\'exico que transforma a string em uma lista de tokens e partir destes
	escrever o parser.
	\item[6 .] (\emph{f\'acil}) --- Defina uma fun\c{c}\~ao
	\begin{center}
		\texttt{statistics :: PGN -> IO()}
	\end{center}
	que coleta e imprime as seguintes informa\c{c}\~oes sobre os jogos contidos no arquivo PGN processado:
	\begin{itemize}
		\item Quantos jogos foram disputados?
		\item Quantas partidas foram vencidas pelas pe\c{c}as brancas, e pelas pe\c{c}as pretas? 
		\item Em cada jogo, qual a quantidade de pe\c{c}as que cada jogador possui ao final da partida?
	\end{itemize}
	Para simplificar o processo de teste, defina uma fun\c{c}\~ao para cada uma destas estat\'isticas.
\end{itemize}

\subsection{Descrevendo a Situa\c{c}\~ao Atual do Tabuleiro}

Toda a nota\c{c}\~ao apresentada at\'e o momento serve apenas para descri\c{c}\~ao de uma sequ\^encia de movimentos 
que define um jogo completo de xadrez. Por\'em, em alguns tipos de quebra-cabe\c{c}as \'e \'util descrever a 
situa\c{c}\~ao atual do tabuleiro de maneira expl\'icita\footnote{Um tipo destes quebra-cabe\c{c}as consiste em 
dada uma configura\c{c}\~ao de tabuleiro um dos lados (preto ou branco) deve obter um xeque-mate em 
$n\geq 1$ movimentos.}.
A nota\c{c}\~ao FEN --- Forsyth-Edwards Notation --- \'e utilizada para este prop\'osito. Uma descri\c{c}\~ao
detalhada sobre este formato pode ser encontrada em:

\begin{center}
	\href{http://en.wikipedia.org/wiki/Forsyth-Edwards\_Notation}
	     {http://en.wikipedia.org/wiki/Forsyth-Edwards\_Notation}
\end{center}
Ao contr\'ario do formato utilizado em arquivos PGN este \'e bem simples e pode ser encontrado de duas formas:
\begin{itemize}
	\item Em um arquivo texto de extens\~ao \texttt{.fen}.
	\item Em uma tag de nome \texttt{FEN} em um arquivo PGN. Neste caso, este arquivo possuir\'a um tag de nome 
	      \texttt{FEN} e respectivo valor uma string que descreve o estado atual do tabuleiro. 
\end{itemize}

\begin{itemize}
	\item [7. ] (\emph{f\'acil}) --- Descreva utilizando uma gram\'atica livre de contexto, o formato FEN.
	\item [8. ] (\emph{m\'edio}) --- Descreva a nota\c{c}\~ao FEN utilizando um tipo de dados Haskell. Defina
	            uma inst\^ancia da classe \texttt{Show} para seu tipo definido de maneira que a string produzida
	            seja similar a uma visualiza\c{c}\~ao de um tabuleiro de xadrez.
	\item [9. ] (\emph{dif\'icil}) --- Defina a fun\c{c}\~ao
	\begin{center}
		\texttt{parseFEN :: Parser Char FEN}
	\end{center}
	que realiza o parsing de uma string que representa a nota\c{c}\~ao FEN. Novamente, voc\^e pode optar por
	utilizar um analisador l\'exico ao inv\'es de realizar o parsing diretamente. Modifique o parser de arquivos
	PGN de maneira que este reconhe\c{c}a tags FEN e retorne o tipo de dados PGN (j\'a definido por voc\^e) e o
	tipo de dados FEN.
\end{itemize}

\subsection{Especifica\c{c}\~ao da Intera\c{c}\~ao}

At\'e o presente momento n\~ao foi especificado como dever\'a ser a intera\c{c}\~ao do usu\'ario de seu software.
Como j\'a especificado anteriormente, seu programa dever\'a processar dois tipos de arquivos: PGN e FEN.
\begin{itemize}
	\item [10.] (\emph{m\'edio}) --- Elabore uma interface de linha de comando para permitir ao usu\'ario selecionar
	            qual o tipo de arquivo ele deseja processar. Preferencialmente, utilize uma estrutura similar a 
	            par\^ametros passador a comandos executados em um terminal.
	\item [11. ] (\emph{f\'acil}) --- Produza como sa\'ida para arquivos PGN suas estat\'isticas e para arquivos FEN
	            um desenho do tabuleiro. Caso o arquivo PGN possua uma tag FEN, primeiro imprima o tabuleiro e depois
	            apresente as estat\'isticas.
	\item[12. ] (\emph{m\'edio}) --- Altere o seu parser de arquivos PGN para permitir que estes arquivos 
	      possuam coment\'arios.
	\item[13. ] (\emph{dif\'icil}) --- Os tipos de dados PGN e FEN s\~ao complementares. Enquanto um descreve os movimentos,
	      o outro descreve a situa\c{c}\~ao atual do tabuleiro. Utilizando estes dois tipos de dados \'e poss\'ivel
	      descrever uma fun\c{c}\~ao que \'e capaz de executar um jogo de xadrez ``passo-a-passo'', isto \'e, a
	      partir de cada movimento descrito no tipo PGN pode-se modificar o tabuleiro descrito pelo tipo FEN.
	      Implemente uma fun\c{c}\~ao que dado um valor de tipo PGN permita repetir todo um jogo mostrando passo-a-passo
	      o estado do tabuleiro. Note que um efeito colateral desta fun\c{c}\~ao \'e a descoberta de movimentos ilegais
	      descritos em um arquivo PGN.
\end{itemize}
 
\end{document}


